Understanding that different localities may have unique processes and requirements, the \textbf{iBrief} application offers a flexible and customizable \gls{workflow} management system. This system empowers administrators to tailor the issue resolution process to their specific needs. Here's how the customization features work:

\begin{itemize}
    \item \textbf{Workflow Templates:} The application comes with a set of predefined workflow templates that can serve as starting points for customization. Administrators can select a template that closely matches their process and make adjustments as needed.
    
    \item \textbf{Drag-and-Drop Interface:} A user-friendly drag-and-drop interface allows administrators to define and modify the stages of the issue resolution process. They can add, remove, or reorder stages according to their operational needs.
    
    \item \textbf{Custom Statuses and Rules:} Administrators can define custom statuses for issues (e.g., `Awaiting Approval`, `Requires External Help`) and set rules for transitions between these statuses. This includes conditions that must be met for an issue to move from one stage to another.
    
    \item \textbf{Role-Based Actions:} The system allows the mapping of specific actions to user roles. For instance, only certain roles may have the permission to mark an issue as `Resolved` or to escalate it for further review.
    
    \item \textbf{Notifications and Triggers:} Customizable notifications and triggers can be set up to alert users and agents when an issue moves to a new stage or requires specific actions.
    
    \item \textbf{Reporting and Analytics:} Administrators can generate reports to analyze the effectiveness of their customized workflows and make data-driven decisions for further refinement.
\end{itemize}

By leveraging these customizable workflow management features, local authorities can ensure that the \textbf{iBrief} application aligns with their internal processes and meets the unique demands of their community. This level of flexibility is crucial for fostering a responsive and efficient issue resolution environment.
