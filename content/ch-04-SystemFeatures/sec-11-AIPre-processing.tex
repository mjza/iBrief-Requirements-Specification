The \textbf{\gls{AI} \gls{pre-processing}} feature introduces a layer of intelligence to the \textbf{iBrief} application, leveraging artificial intelligence to enhance user inputs, ensure content quality, and optimize the issue reporting process. This sophisticated feature provides several key capabilities:

\begin{itemize}
    \item \textbf{Auto Categorization and Correction:} Upon issue reporting, the \gls{AI} system analyzes the provided description and automatically categorizes the issue into predefined classes, streamlining the process for users. Simultaneously, it can identify and correct common typos or mistakes in the report.
    
    \item \textbf{Auto Suggestions:} Based on historical data and patterns, the \gls{AI} system offers suggestions to users as they describe an issue, speeding up the reporting process and providing users with potential terms or phrases they might be intending to use.
    
    \item \textbf{Content Filtering:} The system actively screens and filters out offensive, inappropriate, or harmful words in user descriptions, ensuring that the content within the application remains respectful and professional.
    
    \item \textbf{Fake Issue Detection:} Leveraging machine learning models, the \gls{AI} system can identify patterns associated with fake issue reporting, flagging these instances for review. This reduces the potential for system misuse and ensures resources are directed towards genuine concerns.
    
    \item \textbf{User Authenticity Verification:} The \gls{AI} system evaluates user patterns and behaviors to identify potential fake or malicious user accounts, allowing administrators to take preventive actions against spam or disruptive activities.
    
    \item \textbf{Predictive Analytics:} By analyzing past data, the \gls{AI} system can predict potential future issues or hotspots within the city, allowing local authorities to take preemptive measures.
    
    \item \textbf{Image Recognition:} When users attach photos to their reports, \gls{AI}-driven image recognition can provide preliminary analysis, identifying common issues in the images (e.g., potholes, graffiti) and further streamlining the categorization process.

    \item \textbf{Analyzing Social Medias:} The \gls{AI} can scan and analyze social media platforms to extract reported issues and complaints, automatically importing relevant data into the \textbf{iBrief} system for further action.
\end{itemize}

The \textbf{\gls{AI} \gls{pre-processing}} feature embodies the commitment of the \textbf{iBrief} application to harnessing cutting-edge technologies to offer users a seamless experience, enhance content quality, and ensure efficient and accurate issue reporting and management.
