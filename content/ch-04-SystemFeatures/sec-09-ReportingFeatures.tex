The \textit{Reporting Features} of the \textbf{iBrief} application are central to its functionality, providing users, local authorities, and other stakeholders with valuable insights into issues within cities. These features are designed to ensure that issues are documented thoroughly, tracked efficiently, and addressed effectively.

\begin{itemize}
    \item \textbf{Interactive Issue Reporting:}
    Users can submit comprehensive issue reports, including textual descriptions, media attachments (photos or videos), and precise GPS location tagging, ensuring each report's clarity and accuracy.
    
    \item \textbf{Categorized Issue Reporting:}
    To streamline the resolution process, users can categorize their issues, allowing local authorities to prioritize and address them based on their nature and urgency.
    
    \item \textbf{User Feedback and Voting System:}
    Users can vote on issues reported by others, elevating their visibility and priority. This fosters a community-driven approach to issue resolution.
    
    \item \textbf{Historical Data Review:}
    The application retains a history of reported issues, allowing users and authorities to review past reports, track patterns, and implement preventive measures.
    
    \item \textbf{Real-time Progress Tracking:}
    Once issues are reported, users can monitor their status in real-time, ensuring transparency and keeping them informed of resolution efforts.
    
    \item \textbf{Automated Reporting Summaries:}
    For stakeholders like local authorities and support teams, the application offers automated report generation features, summarizing issue patterns, resolution rates, and other relevant metrics to aid decision-making.
\end{itemize}

The \textit{Reporting Features} of the \textbf{iBrief} application emphasize transparency, community engagement, and efficient problem resolution, ensuring a proactive approach to urban issue management.
