While the \textbf{iBrief} application does not anticipate migrating data from old systems, it is essential to establish guidelines for any future data migration scenarios. For example, in the following scenarios:

\begin{enumerate}
    \item If it is needed to move data between environments (e.g., from development to production).
    \item If it is needed to plan future integration with other systems which might require data transfers.
    \item If it is needed to anticipate potential platform or infrastructure changes in the future that would necessitate data movement.
    \item If there is a possibility of consolidating or restructuring data within the same environment.
    \item For backup and disaster recovery scenarios where data might need to be restored from backups.
    \item If it is needed to consider geographic expansions and might have to replicate or move data to servers in other regions for lower latency.
\end{enumerate}

Therefore, data migration can arise from various operational needs or strategic decisions, and having a clear set of requirements ensures that data integrity, security, and availability are maintained throughout the migration process.

\begin{enumerate}
    \item \textbf{Data Integrity:}
    \begin{itemize}
        \item Any data migration process must ensure that data remains consistent before and after the migration. There should be no data loss, and data quality should be maintained.
    \end{itemize}
    
    \item \textbf{Security:}
    \begin{itemize}
        \item Data migrations should follow strict security protocols. Any data in transit should be encrypted, and access to data during the migration process should be limited to authorized personnel.
    \end{itemize}
    
    \item \textbf{Migration Testing:}
    \begin{itemize}
        \item Before finalizing any migration, a test migration should be conducted to ensure that the process works as expected and that data integrity is maintained.
    \end{itemize}
    
    \item \textbf{Backup and Recovery:}
    \begin{itemize}
        \item Prior to migration, comprehensive backups of the data should be taken. In case of any issues during the migration process, there should be a clear rollback plan to restore the data from backups.
    \end{itemize}
    
    \item \textbf{Documentation:}
    \begin{itemize}
        \item All migration processes should be well-documented, detailing the steps involved, tools used, personnel involved, and any issues encountered.
    \end{itemize}
    
    \item \textbf{Downtime Management:}
    \begin{itemize}
        \item If the migration process requires any system downtime, this should be communicated in advance to relevant stakeholders. Efforts should be made to minimize this downtime.
    \end{itemize}
\end{enumerate}

By establishing clear data migration requirements, the \textbf{iBrief} application is positioned to handle any future data migration scenarios efficiently and securely, ensuring continuous service availability and data reliability.
