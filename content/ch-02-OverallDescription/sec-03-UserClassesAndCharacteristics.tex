The \textbf{iBrief} application is designed to cater to a diverse set of users, each having distinct roles and responsibilities. Here are the primary user classes and their characteristics:

\begin{itemize}
    \item \textbf{End Users (Citizens):} These are general users, often city residents, who utilize the application to report issues they encounter in their surroundings. They might not necessarily have technical expertise but are critical to the system as they serve as the primary data sources.

    \item \textbf{Officer Users (City Officials):} Representatives from local authorities, these users oversee the reported issues, prioritizing and delegating them to appropriate agents for resolution. They also liaise with other government agencies if required.

    \item \textbf{Agents (Field Officers):} These are on-ground personnel assigned to investigate, validate, and rectify the reported issues. Their in-depth local knowledge coupled with the application's tools ensures efficient issue resolution.

    \item \textbf{Administrators:} These are high-tier users, often from the city's IT department or the software provider, who manage the system's \gls{back-end}. They handle user permissions, system configurations, updates, and ensure the application runs seamlessly.

    \item \textbf{Support Staff:} Personnel responsible for assisting users with any application-related queries or issues. They bridge the gap between end-users and technical teams, ensuring smooth user experiences.

    \item \textbf{Decision Makers (City Planners/Managers):} While not daily users, this group leverages the data and insights generated by the application for broader city planning, resource allocation, and policy-making.

    \item \textbf{Community Leaders:} Individuals or groups who, while might not be directly involved in issue resolution, use the platform to gain insights into community needs and advocate for change.

\end{itemize}

While the application is robust and packed with features, its design philosophy emphasizes user-friendliness. This ensures that individuals, regardless of their technical acumen or digital proficiency, can navigate and utilize the platform effectively.

