The development and successful deployment of the \textbf{iBrief} application hinge on several assumptions and dependencies:

\begin{itemize}
    \item Users possess smartphones equipped with active internet connectivity.
    \item In the absence of internet access, users can record issues, which are stored locally and synced to the server once online connectivity is restored.
    \item Local governing bodies and authorities are amenable to collaboration and possess the option to integrate their existing systems with the application. Alternatively, they can utilize the \gls{dashboard application} for oversight and management of reported concerns.
    \item The project receives sufficient resources spanning development, testing, and operations.
    \item Reliance on external \gls{API}s and web services, particularly for geolocation data and push notifications, necessitates their consistent availability and reliability.
    \item The application's infrastructure is scalable to accommodate fluctuations in user traffic and data volume.
    \item Continuous updates and maintenance are essential for adhering to evolving security standards and addressing potential vulnerabilities.
    \item User feedback mechanisms are in place, ensuring that the application evolves in alignment with user needs and expectations.
\end{itemize}
