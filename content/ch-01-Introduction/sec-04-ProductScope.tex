\subsection{Project Overview and Description}
    The \textbf{iBrief} application serves as an integrated digital platform, empowering citizens to promptly report urban issues to local authorities. Aimed at enhancing public infrastructure management and community welfare, this initiative simplifies the reporting, tracking, and resolution processes. Although the application's current moniker is `iBrief', potential rebranding and white labeling might occur for marketing strategies. 

\subsection{Objectives and Features}
    The \textbf{iBrief} application's core objectives encompass:
    \begin{enumerate}[label=\alph*)]
        \item Offering an intuitive mobile platform for incident reporting.
        \item Ensuring traceability from issue submission to resolution.
        \item Facilitating seamless communication between citizens and local authorities.
        \item Augmenting urban safety, cleanliness, and overall living standards.
        \item Widening accessibility across iOS and Android users.
    \end{enumerate}

    To realize these objectives, \textbf{iBrief} will incorporate:
    \begin{itemize}
        \item User-centric account management.
        \item Simplified issue reporting with \gls{GPS} location tagging.
        \item Support for multimedia attachments.
        \item Real-time issue tracking and notifications.
        \item Integration with local authority systems.
        \item A comprehensive \gls{web-based} administrative dashboard.
    \end{itemize}

\subsection{Deliverables and Scope}
    The key project deliverables include:
    \begin{enumerate}
        \item A \gls{iOS} \gls{client application}.
        \item A robust \gls{Android} counterpart.
        \item A \gls{React}-powered administrative dashboard.
        \item A secure, \gls{AWS}-hosted back-end infrastructure.
    \end{enumerate}
    
    The project's ambit covers the creation and deployment of mobile applications, the \gls{web-based} dashboard, and the foundational back-end. This also entails the system's integration with existing local authority infrastructures for issue resolution efficacy.
    
\subsection{Stakeholders and Dependencies}
    Key project stakeholders comprise:
    \begin{itemize}
        \item End-users or citizens.
        \item Local governing bodies.
        \item Project overseers and sponsors.
        \item Development and \gls{QA} personnel.
    \end{itemize}
    
    The project hinges on:
    \begin{itemize}
        \item Gaining access to local authority data and systems.
        \item Ensuring stakeholder collaboration and approvals.
        \item Procuring necessary development and testing resources.
    \end{itemize}

\subsection{Assumptions, Risks, and Change Management}
    The project's foundational assumptions include:
    \begin{itemize}
        \item Users possessing internet-enabled smartphones.
        \item Collaborative disposition of local authorities.
        \item Adequate resource allocation.
    \end{itemize}

    Foreseen project risks encompass:
    \begin{itemize}
        \item System integration hurdles.
        \item Varied user engagement levels.
        \item Data privacy and security challenges.
        \item Potential technical roadblocks.
    \end{itemize}
    
    A structured change control process will be in place, mandating documented changes to undergo impact assessments and secure approvals from key stakeholders before execution.
