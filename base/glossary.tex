\newglossaryentry{iOS}{
  name={iOS},
  description={Apple's mobile operating system used for its iPhone, iPad, and iPod Touch devices}
}

\newglossaryentry{Android}{
  name={Android},
  description={An open-source mobile operating system developed by Google}
}

\newglossaryentry{React}{
  name={React},
  description={A JavaScript library for building user interfaces, maintained by Facebook}
}

\newglossaryentry{AWS}{
  name={AWS},
  description={Amazon Web Services, a subsidiary of Amazon providing cloud computing platforms and \gls{API}s}
}

\newglossaryentry{client application}{
  name={client application},
  description={A software application that runs on the client side, typically on a user's device, and connects to a server to fetch or send data}
}

\newglossaryentry{dashboard application}{
  name={dashboard application},
  description={An application interface that provides users with an overview and control of specific metrics or features}
}

\newglossaryentry{back-end}{
  name={back-end},
  description={The server-side of an application, responsible for data storage, retrieval, and processing}
}

\newglossaryentry{front-end}{
    name={front-end},
    description={Front-end, often referred to as the "client-side," is the user interface and presentation layer of a software application or website. It encompasses the visual and interactive elements that users directly interact with, such as web pages, mobile app interfaces, and graphical user interfaces (GUIs). Front-end development involves designing and building the user interface, ensuring a seamless and user-friendly experience. Front-end technologies may include HTML, CSS, JavaScript, and various libraries and frameworks that enable the rendering of content, responsive design, and user interactivity. Front-end development plays a critical role in delivering a visually appealing and intuitive user experience}
}

\newglossaryentry{end-user}{
    name={end-user},
    description={An end-user, often referred to as a "user," is an individual or entity who directly interacts with a software application, system, or product to perform specific tasks or achieve particular objectives. End-users are the ultimate consumers of the product and may include individuals, organizations, or customers. They utilize the software or product to accomplish their intended goals, and their experience and satisfaction are vital considerations in the design and functionality of the system. The term "end-user" is commonly used in software development and user experience (UX) design to emphasize the importance of meeting the needs and expectations of those who will use the product in real-world scenarios}
}

\newglossaryentry{web-based}{
    name={web-based},
    description={Web-based refers to applications, services, or systems that are accessible and operated through a web browser over the internet or an intranet. These applications are hosted on remote servers and do not require the installation of additional software on the user's device. Users can access web-based applications from various devices, including computers, smartphones, and tablets, by simply entering a URL in their web browser. Web-based solutions are known for their accessibility, platform independence, and the ability to provide real-time data and collaboration among users from different locations. They have a wide range of applications, from web-based email clients to online productivity tools and cloud-based software services}
}

\newglossaryentry{UI}{
  name={UI},
  description={User Interface, the space where user interactions occur with a digital product or service}
}

\newglossaryentry{UX}{
    name={UX},
    description={User Experience (UX) refers to the overall experience that an individual has while interacting with a product, system, or service, typically in the context of digital technology. It encompasses a user's perceptions, emotions, and responses when using a product or system. UX design aims to create products that are intuitive, user-friendly, and enjoyable, by considering factors such as usability, accessibility, visual design, and user satisfaction. A positive UX enhances user engagement and satisfaction, ultimately leading to more successful and effective digital experiences}
}

\newglossaryentry{AI}{
  name=AI,
  description={Short for Artificial Intelligence, AI refers to the simulation of human intelligence in machines, especially computer systems. These processes include learning (the acquisition of information and rules for using the information), reasoning (using the rules to reach approximate or definite conclusions), and self-correction. AI technologies include machine learning, neural networks, and natural language processing among others.}
}

\newglossaryentry{QA}{
    name={QA},
  description={Quality Assurance, the process of ensuring product quality}
}

\newglossaryentry{GPS}{
    name={GPS},
  description={Global Positioning System, a satellite-based navigation system}
}

\newglossaryentry{URL}{
    name={URL},
  description={Uniform Resource Locator, a web address}
}

\newglossaryentry{API}{
    name={API},
  description={Application Programming Interface, a set of rules for software communication}
}

\newglossaryentry{mobile application}{
    name={mobile application},
    description={Mobile applications, often referred to as "mobile apps," are software programs specifically designed and developed for use on mobile devices, such as smartphones and tablets. These applications offer a wide range of functionalities, including but not limited to productivity, entertainment, communication, and utility. Mobile apps are typically available for download and installation from mobile app stores, allowing users to access and utilize them on their portable devices. They enhance the functionality and user experience of mobile devices by providing tailored and user-friendly solutions for various purposes}
}

\newglossaryentry{QR-code}{
    name={QR-code},
    description={A QR code, short for Quick Response code, is a two-dimensional barcode that contains encoded information. It can store various types of data, such as text, URLs, contact information, or other binary data. QR codes are often used for quick and efficient data retrieval, typically by scanning the code with a smartphone or QR code reader. They have widespread applications in marketing, ticketing, inventory tracking, and many other fields, providing a convenient way to access digital content or perform actions by simply scanning the code with a compatible device}
}

\newglossaryentry{Kanban board}{
    name={Kanban board},
    description={A Kanban board is a visual project management tool used to visualize work processes, tasks, and workflow in a visual format. It typically consists of columns representing different stages of work, such as "To Do," "In Progress," and "Completed." Tasks or work items are represented as cards or sticky notes and are moved through the columns as they progress in the workflow. Kanban boards are widely used in agile and lean methodologies to improve efficiency, track work progress, and optimize task management}
}

\newglossaryentry{GDPR}{
    name={GDPR},
    description={General Data Protection Regulation (GDPR) is a regulation introduced by the European Union (EU) in 2018 to strengthen and unify data protection for individuals within the EU. It offers citizens greater control over their personal data and simplifies the regulatory environment for international businesses. Key provisions include data subject rights, data breach notifications, and stringent processing requirements}
}

\newglossaryentry{CCPA}{
    name={CCPA},
    description={California Consumer Privacy Act (CCPA) is a data privacy law that took effect in California, USA, in 2020. It grants California residents new rights regarding how their personal information is collected, used, and sold. The CCPA provides individuals with the right to know, the right to delete, and the right to opt-out of the sale of personal data. It's seen as one of the most stringent data protection measures in the U.S., drawing comparisons to the EU's GDPR}
}

\newglossaryentry{PIPA}{
    name={PIPA},
    description={Personal Information Protection Act (PIPA) is an Alberta's privacy legislation for the private sector, which governs the collection, use, and disclosure of personal information by private sector organizations in Alberta}
}

\newglossaryentry{FOIP}{
    name={FOIP},
    description={Freedom of Information and Protection of Privacy Act (FOIP) is an Alberta's privacy legislation for public bodies, which has its own set of rules for how these bodies must handle personal information}
}

\newglossaryentry{PIPEDA}{
    name={PIPEDA},
    description={Canada has its federal privacy legislation known as the Personal Information Protection and Electronic Documents Act (PIPEDA), which sets the ground rules for how businesses must handle personal information in the course of their commercial activity}
}

\newglossaryentry{Microservices}{
  name={Microservices},
  description={A software architectural style that structures an application as a collection of loosely coupled, independently deployable services. These services are small, highly maintainable, and encapsulate specific business functionalities}
}

\newglossaryentry{load balancing}{
  name={load balancing},
  description={A technique used to distribute incoming network traffic across multiple servers to ensure no single server is overwhelmed with too much traffic, thus optimizing responsiveness and maximizing throughput}
}

\newglossaryentry{pre-processing}{
  name={pre-processing},
  description={A data processing technique in which the data is prepared and cleaned before being used in further analysis or operation. In the context of AI, pre-processing may involve normalization, transformation, and the removal of noise or outliers}
}

\newglossaryentry{scalability}{
  name={scalability},
  description={The capability of a system, network, or process to handle an increasing amount of work or its potential to be enlarged in order to accommodate that growth}
}

\newglossaryentry{backup}{
  name={backup},
  description={A copy of data taken and stored elsewhere so that it may be used to restore the original after a data loss event}
}

\newglossaryentry{recovery}{
  name={recovery},
  description={The process of restoring and validating data from a backup after a data loss event, ensuring the integrity and consistency of the restored data}
}

\newglossaryentry{stakeholders}{
  name={stakeholders},
  description={Individuals, groups, or organizations that have a direct or indirect interest in a project or system. Stakeholders can influence or be influenced by the project's actions, objectives, and policies}
}

\newglossaryentry{workflow}{
  name={workflow},
  description={An orchestrated and repeatable pattern of business activities enabled by the systematic organization of resources into processes that transform materials, provide services, or process information}
}
